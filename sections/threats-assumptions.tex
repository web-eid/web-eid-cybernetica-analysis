\newpage
\section{Assumptions that can be satisfied with current technology} 
\label{sec:satisfiable_assumptions}
 
\subsection{Key length} 
We assume that the key lengths are sufficient to prevent brute force attacks.
 
\subsection{Keys are randomly generated} 
We assume that the keys are randomly generated by using high quality randomness that is suitable for cryptographic operations. Therefore  we assume that the keys can not be predicted or guessed. 

\subsection{ID-card keys are generated in the card} 
We assume that the private keys of the ID-card never leave the card and are generated inside of the chip.
 
\subsection{ID-card private keys do not leak}
We assume that the private keys of the ID-card are protected by special hardware and can not be extracted and leaked.

\subsection{ID-card keys can not be deleted} 
We assume that an attacker is not able to delete or regenerate the key pairs of an ID-card.

\subsection{Only strong cryptosystems with sufficient key lengths are used} 
We assume that only strong and standardized cryptosystems are used, which can not be broken with classical computers. E.g., we assume that it is not possible to deduce the private key from the public key in case the key pair was randomly generated.

What cryptosystems can be considered strong in near term (10 years)? In order to answer this question, we rely on the ECRYPT CSA 2018 report~\cite{ECRYPT-CSA}. An excerpt from the Table 4.6 of the report is presented in Table~\ref{tab:keylengths}.

\begin{table}[ht]
    \begin{center}
        \caption{ECRYPT CSA recommended key lengths}
        \label{tab:keylengths}
        \begin{tabular}{|c|c|}
            \hline
            Cryptosystem    & Minimal output/key length\\
            \hline
            \hline
            Symmetric (AES) & 256 \\
            RSA             & 3072\\
            ECC             & 256\\
            Hash function   & 256\\
            \hline
        \end{tabular}
    \end{center}
\end{table}

Concerning padding, ECRYPT CSA report~\cite{ECRYPT-CSA} recommends using RSA-PSS instead of PKCS\#1 v1.5 for new deployments of RSA signature scheme. PSS scheme is also standardised for use in JSON Web Algorithms RFC7518~\cite{RFC7518}. Note that the standard only supports the schemes utilising SHA2-256, SHA2-384 and SHA2-512 hash functions.

In case of elliptic curve signature schemes, there is no need for a special padding. Care must be taken that the bit lengths of the hash function output and ECC signature input match. RFC7518~\cite{RFC7518} specifies three pairs of ECDSA signature scheme curves and hash functions:
\begin{itemize}
    \item ECDSA using P-256 and SHA2-256,
    \item ECDSA using P-384 and SHA2-384, and
    \item ECDSA using P-521 and SHA2-512.
\end{itemize}

\subsection{Quantum computers are not available} 
We assume that large quantum computers, which could break the modern asymmetric cryptosystems are not available to the attacker. In 2019 Quantum Threat Timeline Report\footnote{\url{https://globalriskinstitute.org/publications/quantum-threat-timeline/}} was published and it contains estimates by 22 experts on the field. Half of them estimated that  a quantum computer capable of breaking RSA-2048 will be built within 15 years\footnote{\url{https://quantumcomputingreport.com/our-take/how-many-years-until-a-quantum-computer-can-break-rsa-2048/}}. 

\subsection{Attacker with superuser access has complete access} 
In case an attacker has infected client's device and has superuser access, we assume that the attacker has complete access. Thus, the attacker can choose what to draw on the screen, which values to send to the card reader and the ability to read the keystrokes of the user.

\subsection{Collision resistance property can not be broken} 
We assume that the attacker is not able to break the collision resistance property of the cryptographic hash function that is used for signing.


\subsection{Second preimage resistance property can not be broken} 
We assume that the attacker is not able to break the second preimage resistance property of the cryptographic hash function that is used for signing.

\subsection{The authentication challenge can not be guessed or predicted}
We assume that the authentication challenge is generated randomly and has sufficient length / entropy.
 
\subsection{Session cookie is not predictable} 
We assume that the session cookie has sufficient entropy and is not predictable.

\subsection{Communication channel is protected by TLS}
The communication channel is secured using TLS 1.2 or a newer TLS version. We do not consider attacks against such channels to be in the scope of this analysis as currently these attacks are not feasible without either compromising the client or the server.

\subsection{Secondary channel to inform the user about card use}
At the first sight this seems easy -- just send the user an email, SMS, push notification or alike. On the other hand, there may be some scenarios where this may be discouraged (most notably I-voting where strong evidence of signature device use can be used for vote selling). In case informing the user would be mandatory, the service would have to be centralized as all service providers would not be able to set up the secondary channel. However, by having a centralized service, the privacy of the user becomes an issue. Therefore, the architecture of the functionality has to be well thought through before a decision is made to implement it.








\newpage
\section{Assumptions that can not be satisfied with current technology}
\label{sec:unsatisfiable_assumptions}


\subsection{Card readers with PIN pad and trusted preview}

Even the officially recommended smart card reader with PIN-firewalled PIN pad (Gemalto IDBridge CT710\footnote{\url{https://gemcard.ro/wp-content/uploads/2016/11/Gemalto_IDBridgeCT700_CT710_brochure.pdf}}) seems to be unavailable in Estonia. Requesting a trusted preview panel seems even more infeasible. Thus, the first step towards not having to trust the computing device would be to make the PIN-firewalled PIN pads available to the end-users. By using PIN firewalled smart card readers the complexity of an attack would increase. As a next step, it should be researched whether smart card readers with trusted preview are available on the market and how they could be integrated to the current software ecosystem.

\subsection{Token Binding}
Token Binding (or an equivalent technology) would mitigate a lot of attacks. However, it does not seem to be universally available across all the browsers and platforms.

\subsection{Browser extension can access details of TLS connection}
Some of the proposed countermeasures for the MITM attack relay on the possibility of accessing the public key of the service provider. Currently only Firefox allows an extension to query information about the TLS connection (e.g., the public key of the service provider for the current session).


\subsection{Using separate key pairs for authentication, encryption and authorization}
\label{subsec:separatekeypairs}
Currently, the Estonian ID-card contains two key pairs, out of which the second key pair is only used for issuing legally binding signatures. However, the first key pair is used for multiple functionalities like authentication, encryption/decryption, logging into services, opening VPN connections, etc. This is not the best design as mixed usage scenarios may cause unexpected vulnerabilities.

When the ID-card owner uses his/her card for a longer-period service, say to log onto a terminal, the card has to be in the reader while the user is logged in, and only when the card is removed from the reader, the user is logged out. The current Open eID architecture relies on mutually authenticated TLS. Therefore, Open eID design allows the user to enter PIN1 only once during the authentication phase, after which the card's authentication security environment is left open. Such design allows to use mutually authenticated TLS without constantly asking the user to enter the PIN1. The downside of this design is that other applications running on the same device could also use the first key pair while the authentication security environment on the card is open (in the above example, during the whole period of being logged onto the terminal).

However, the Web eID architecture does not rely on mutually authenticated TLS channels. Thus, it is no longer necessary to keep the card's authentication security environment open after the authentication has been completed. Closing the access to the authentication security environment would prevent other applications from using the private key from the first key pair. Still, access to the card's authentication security environment can not be simply reset, since this can interfere with other legitimate applications that are using that security environment at the same time.

It is an open question to find out how many users would be negatively affected when access to card's authentication security environment would be reset by Web eID. In the long run, it is advised to issue separate key pairs for separate functionalities like authentication, encryption and authorization. With separate key pairs it would be easier to avoid problems arising from different security requirements of different usage scenarios.

