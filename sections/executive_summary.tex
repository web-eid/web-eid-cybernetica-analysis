\chapter{Executive summary}
Estonian State Information Agency (RIA) has initiated a process to update the ID-card (smart card) support in browsers to address various stability and support issues the current TLS client certificate authentication (CCA) based solution has. A team of researchers and developers has created an initial architecture draft for the new solution\footnote{\url{https://github.com/open-eid/browser-extensions2}}, which will be named Web eID. The purpose of the current document is to analyse this architecture from the security point of view.

To do this in a systematic way, we fixed the scope of the analysis, identified all the relevant digital assets present in the architecture, and considered possible actions an attacker could do (Create, Read/Use/Copy, Update, Delete) on them. As a result, this document lists the assumptions that the proposed solution relies on, along with the threats and mitigation measures.

The main scope of the analysis was limited to the components that are in the focus of development during the current project (server application, browser extension and Web eID native application). Note, however, that the full ecosystem is much wider, including the client and server platforms, trust service providers, browsers, etc. There are many potential threats emanating from them, and therefore it should be regularly reviewed if these threats are handled and whether additional mitigation measures should be implemented. 

The biggest change in the Web eID architecture compared to the previously used Open eID is the way how clients are authenticated. It is no longer possible to use client side authentication provided by TLS. While the change adds stability, it also affects the security of authentication. Not all configurations of the new Web eID architecture protect against man-in-the-middle attacks where the attacker is simultaneously able to do DNS-spoofing and provide a valid TLS certificate for client's browser.

Web eID was designed to allow the service provider to select between two protection profiles based on the level of acceptable risk. The first profile is easy to implement and protect, but is vulnerable to the aforementioned powerful man-in-the-middle attack. While in theory the stronger profile provides protection against powerful man-in-the-middle attacks during the authentication phase, it is non-trivial to implement. Currently only Mozilla's Firefox browser provides an API that allows to apply the second protection profile on the client side. Even if implemented, the limitations of browser API-s allow to protect only the authentication phase against the aforementioned man-in-the-middle attacks. Such attacks usually require a certificate authority to be compromised and are therefore unlikely to happen. However, the same risk applies to connections that are being legitimately monitored by corporate proxies.

As the majority of mainstream browsers are built on top of Google's Chromium, Google's representative was contacted in order to inquire about the possibility extend Chromium's API so that the aforementioned man-in-the-middle attacks could be avoided. The representative of Google responded that such measures are not in plan and are most likely not going to be implemented. One of the reasons is that a significant percentage of web traffic goes through corporate middleboxes\footnote{\url{https://blog.cloudflare.com/monsters-in-the-middleboxes/}}, which need to perform man-in-the-middle interception in order to scan traffic~\cite{DBLP:conf/ndss/DurumericMSBSBB17}. Corporate scanning of TLS traffic falls under the class of powerful man-in-the-middle attacks and thus affects Web eID. Therefore, corporate proxies become a single point of failure for Web eID in case the second protection profile is not implemented. An attacker or a compromised employee abusing the corporate proxy can intercept the session tokens and can replace the hash values that are being sent to be digitally signed by the client.

The second downside of the new architecture compared to TLS-CCA lies in the difficulty of preventing session hijacking attacks. TLS-CCA based architecture makes it possible to prevent session hijacking (i.e., copying of session identifiers), but there is no straightforward way to do that in the new architecture. This is a common problem, which is present in most of the mainstream authentication technologies as described in Section~\ref{sec:session_hijacking}. This can be a problem in the following cases:
\begin{itemize}
 \item a vulnerability in the web service or browser gives access to session identifiers,
 \item an attacker has temporary local access and can copy the session identifier,
 \item when HTTPS interception (proxy or middlebox) is used to monitor or mediate the traffic, the session identifier may leak.
 \end{itemize}
